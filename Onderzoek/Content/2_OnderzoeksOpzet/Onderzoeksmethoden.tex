\section{Onderzoeksmethoden}
In dit hoofdstuk zal er worden toegelicht welke onderzoeksmethode gekozen is voor elke deelvraag met daar bij een onderbouwing.
Om de validiteit van het onderzoek te waarborgen is er gebruik gemaakt van \textit{ICT-research methods} \Parencite{HBO-i-reasearch-methods}.

\whitespace
\textit{\textbf{Deelvraag 1:} \SubquestionOne}

\whitespace
Om deze deelvraag te beantwoorden wordt er een \textbf{stakeholdersanalyse} uitgevoerd.
Dit wordt gedaan door samen met de product owner een \textbf{brainstorm} sessie te houden.
Na deze sessie zullen de stakeholders geprioriteerd worden op basis van belang en invloed op het project.
Tot slot worden de stakeholders in een Mendelow matrix \Parencite{MandelowMatrix} geplaatst om hun positie weer te geven in het project.
Aan het einde van de stakeholdersanalyse worden de resultaten terug gelegt aan de product owner om de resultaten te valideren.

\whitespace
\textit{\textbf{Deelvraag 2:} \SubquestionTwo}

\whitespace
Door te onderzoeken hoe de huidige software architectuur in elkaar zit en onderhouden is kan er een beeld geschetst worden van de huidige problemen met Snakware Cloud.
Hierom is er voor gekozen om gebruik te maken van \textbf{IT architecure sketching} om de huidige softwarearchitectuur in beeld te brengen.
Samen met het R\&D team zal er een sessie gepland worden om de huidige architectuur in beeld te krijgen.
Het resultaat dat uit deze deelvraag komt wordt gebruikt ter ondersteuning van deelvraag 3.

\whitespace
\textit{\textbf{Deelvraag 3:} \SubquestionThree}

\whitespace
Om deze deelvraag te beantwoorden wordt er een semi-gestructureerde \textbf{expert interview} gehouden met X-X om er achter te komen wat de huidige knelpunten zijn bij Snakeware cloud.
Na de expert interviews zal er een \textbf{task analyse} uitgevoerd worden om de werkwijze (workflows) in beeld te krijgen.

\todo[inline]{met wie is het beste om het interview te houden janny, hans of johan of allemaal (ben dan alleen bang voor tijd)}
\todo[inline]{ook te kort denk ik}

\whitespace
\textit{\textbf{Deelvraag 4:} \SubquestionFour}

\whitespace
Om deze deelvraag te beantwoorden wordt er gebruik gemaakt van \textbf{explore user requirements}.
De communicatie methode met de stakeholders wordt bepaald op basis van hun positie binnen het project door middel van de Mandelow matrix.
Voor de sleutelfiguren worden er semigestructureerde \textbf{interviews} gehouden om genoeg vrijheid te geven tijdens de gesprekken om dieper op vragen in te gaan.
Daarnaast worden er met de geïnteresseerde een \textbf{Focus group} gepland om hier met de betrekende stakeholders meerdere onderwerpen te bespreken die belangrijk zijn voor het project.
Voor de focus groep zal er gebruik gemaakt worden van een aantal voorbereide vragen om de focus groep in een goede richting te sturen.
Als de eisen en wensen zijn bepaald door middel van de focus group en interviews worden ze genoteerd zodat ze in de volgende deelvraag geprioriteerd kunnen worden.

\whitespace
\textit{\textbf{Deelvraag 5:} \SubquestionFive}

\whitespace
De lijst van requirements uit deelvraag 4 moet nog worden verwerkt om er gebruikt van te kunnen maken.
Dit wordt gedaan door middel van \textbf{requirements prioritization} zal er verschillende prioriteit niveaus toegekend worden aan de requirements.
Deze niveaus worden in beeld gebracht door middel van MoSCoW-methode \Parencite{MoSCoW}.

De prioriteit van de requirements wordt berekend op basis van de invloed en het belang van de stakeholders,
de tevredenheidsscore [1,2,\ldots,5] bij implementatie, de ontevredenheid score [1,2,\ldots,5] als het niet geïmplementeerd wordt en de duur van de implementatie (1,2,3,5,8) die wordt weer gegeven door een relatief getal.
De berekening wordt gedaan door middel van de volgende formule:

\whitespace
\begin{center}
	\(Score = tevredenheidsscore - ontevredenheid score + (9 - duur)\)
\end{center}

\whitespace
nadat er een score is berekend wordt er een prioriteit niveau toegegeven op basis van de MoSCoW methode:

\whitespace
Must have:  \(x \in \mathbb{R} : 14 \leq x \leq 18\)

\whitespace
Should have: \(x \in \mathbb{R} : 14 \leq x \leq 18\)

\whitespace
Could have: \(x \in \mathbb{R} : 14 \leq x \leq 18\)

\whitespace
Won't have: \(x \in \mathbb{R} : 14 \leq x \leq 18\)

\whitespace
De requirement worden vervolgens genoteerd in een gemodificeerde versie van een snow card.
Na het maken van de requirement wordt er terug gekoppeld met de stakeholder om het te verifiëren
als de volledige lijst gemaakt is wordt het gecheckt met de product owner en de bedrijfsbegeleider.

\todo[inline]{Maak fancy snow card en fancy formule en daarna tekst verbeteren scores kloppen ook nog niet want deze moeten besproken worden en eigelijk pas gesteld worden tijden de deelvraag ?}
