\section{Onderzoeksvragen}
Om de doelstelling van het onderzoek te behalen moeten de stakeholders geïdentificeerd worden.
Door middel van de stakeholders kan er een lijst van requirements opgesteld worden.
Hierom is de volgende hoofdvraag opgesteld.

\whitespace
\textit{\textbf{Hoofdvraag:} \MainQuestion}

\whitespace
Om een volledig antwoord te kunnen geven op de hoofdvraag, wordt deze vraag opgedeeld in meerdere deelvragen.
Elke van deze deelvragen wordt afzonderlijk behandeld om een omvattend antwoord te krijgen.

\whitespace
Voor het opstellen van de requirements is het belangrijk om te weten voor wie je het maakt.
Daarom is het belangrijk om de stakeholders van het project in beeld te brengen. 
Hierom is de volgende deelvraag gebruikt:

\whitespace
\textit{\textbf{Deelvraag 1:} \SubquestionOne}

\whitespace
Om de huidige problemen van het Snakeware cloud platform in beeld te brengen is het belangrijk dat er gekeken wordt naar de huidige software-architectuur.
Hier uit wordt een lijst met problemen verzameld die de huidige software-architectuur nu heeft, en wordt ter ondersteuning gebruikt voor deelvraag 3 (\textit{\SubquestionThree}).
Daarom is de volgende deelvraag opgesteld:

\whitespace
\textit{\textbf{Deelvraag 2:} \SubquestionTwo}

\whitespace
Omdat een van de doelen van het  proof of concept is het oplossen van de huidige problemen die de klant en Snakeware heeft met het huidige systeem.
Daarom is het belangrijk om te inventariseren wat de huidige knelpunten zijn van het systeem. 
Hierom is de volgende deelvraag gemaakt:

\whitespace
\textit{\textbf{Deelvraag 3:} \SubquestionThree}

\whitespace
Om het systeem te kunnen ontwikkelen moeten er requirements aan het systeem gesteld worden.
Deze requirements moeten op basis van de eisen en wensen van de stakeholders gemaakt worden.
Hier uit zal een lijst requirements komen waar mee het systeem gerealiseerd wordt.
Daarom is de volgende deelvraag gemaakt:

\whitespace
\textit{\textbf{Deelvraag 4:} \SubquestionFour}

\whitespace
Nadat er een lijst van requirements zijn opgesteld als resultaat van deelvraag 3.
Deze lijst is nog niet handig om te gebruiken omdat de lijst niet geprioriteerd is.
Om deze lijst te prioriteteren wordt de volgende deelvraag gesteld.

\whitespace
\textit{\textbf{Deelvraag 5:} \SubquestionFive} \\
