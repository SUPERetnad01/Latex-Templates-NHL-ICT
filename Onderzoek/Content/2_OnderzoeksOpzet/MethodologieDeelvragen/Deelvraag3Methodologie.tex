\subsection{Deelvraag 3: Knelpunten}
Een van de doelen van het proof of concept is het oplossen van de huidige problemen die de klant en Snakeware nu hebben met het huidige systeem.
Daarom is het belangrijk om de huidige knelpunten van het systeem te inventariseren.
Hierom is de volgende deelvraag gemaakt:

\begin{center}
    \textit{\SubquestionThree}
\end{center}

\whitespace[0.2]
Om deze deelvraag te beantwoorden wordt er een semigestructureerd \textbf{expertinterview} gehouden.
Binnen Snakeware zijn er meerdere mensen die geschikt zijn om de knelpunten van het Snakeware Cloud platform te kunnen aankaarten.
Bij de volgende mensen worden de interviews afgenomen:
\begin{itemize}
    \item[-] Janny Reitsma (Service desk lead): Janny heeft veel inzicht in waar de huidige klanten van Snakeware tegen aanlopen.
        Verder krijgt ze alle klachten van de klanten van Snakeware mee en weet ze waar de huidige klanten van Snakeware behoefte aan hebben.
    \item[-] Rob Douma (Product owner van meerdere projecten): Rob werkt aan meerdere projecten als product owner en weet veel van Snakeware Cloud.
        Hij heeft veel technische kennis over het platform en kan goed in beeld brengen wat de huidige technische limitaties zijn van het platform.
\end{itemize}
%
% \whitespace
% Er is overwogen om Hans Hoomans (CEO) en Johan Nieuwehuis (CTO) te interviewen om de huidige knelpunten in beeld te brengen.
% Dit is uiteindelijk niet gedaan vanwege de tijd die beschikbaar is voor het onderzoek.
% Hierdoor mist er een stukje toekomst visie van het resultaat.
