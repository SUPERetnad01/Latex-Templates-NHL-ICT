\section{Rob Douma Interview}

\subsection*{Vragen}
\begin{itemize}
    \item{Introductie}
    \begin{itemize}
        \item[1]{Wie ben je?}
        \item[2]{Wat doe je binnen Snakeware?}
    \end{itemize}
    \item{Profiel van de klanten}
    \begin{itemize}
        \item[3]{Wie gebruikt nu het CMS zijn dat mensen in de winkels of systeembeheerders?}
    \end{itemize}\item{Problemen met het CMS}
    \begin{itemize}
        \item[4]{Zijn er problemen waar jij nu tegen aanloopt?}
        \item[5]{Wat zijn de grootste technische limitaties waar je tegenaan loopt}
        \item[6]{Als je een fundamentele aanpassing kon maken aan het huidige CMS wat zou dat zijn?}
    \end{itemize}\item{Positieve kanten van het huidige CMS}
    \begin{itemize}
        \item[1]{Wat vind je nu fijn aan het huidige CMS?}
        \item[2]{Als er een punt van het CMS meer aandacht aan besteed kon worden wat zal dat zijn?}
    \end{itemize}
\end{itemize}


\subsection*{Transcriptie}

\DanteInt{Misschien even een klein voorstel rondje. Wie ben je en wat doe je binnen Snakware.}

\RobInt{Ik ben Rob Douma vroeger was ik informatieanalisten tegenwoordig lijkt dat meer op product owner.
	Tegenwoordig is het iets breder dan dat, ik doe ook wel wat met database inrichting querys en dat soort vragen.
	Ik heb ook wel een klein beetje kennis van de codes soms doe ik zelf wat in de code.
	Dus aan de ene kant zit ik dus een beetje aan de voorkant van het traject dus dat probeer ik nou In de designfase / analyse fase zeg maar het afkaderen van de oplossing.
	Ja aan de andere kant ben je dan ook betrokken dus gedurende de technische realisatie om daar zo ondersteuning te bieden
}

\DanteInt{Voor de twee belangrijkste vragen wat vind je nu fijn aan het huidig CMS en wat zijn de huidige negatieve kanten van het CMS.
	Daarnaast wat is het huidige klanten profiel van een gebruiker van het CMS.
}

\RobInt{Dit hangt van de klant af, in het geval van Dormio is het de marketing afdeling.
	Zij zetten de teksten neer die bij de huisjes horen en de content er om heen.
	Het zal dus met name de pers redactie of marketing afdeling zijn van de klanten.
}

\DanteInt{Met de problemen rond het huidige CMS zijn er specifieke technische limitaties waar je nu tegenaan loopt.
	Waarbij je het nu niet meer doet omdat het niet technische mogelijk is binnen het CMS.}

\RobInt{Nou stukje flexibiliteit qua het groeperen van content dat mis ik wel een beetje.
	Met andere woorden stel je voor je wilt een aantal artikelen in een slider hebben tonen dan heb je daar eigenlijk soort van containertje voor nodig.

	\whitespace
	Dat containertje kan gebruikt woorden in de frontend dan weet dat de content in de slider moet worden getoond. (\textbf{Dante Klijn:} O het afbakenen van artikelen in specifieke plekken)
	We hebben in secties dus dat dat lijkt er na al een beetje op.
	Het nadeel van secties is dat je die vast zet op de pagina dus die heeft een vaste positie.
	Bijvoorbeeld als je een slider hebt je kan die dan altijd onder op de pagina laten staan.
	Dus eigenlijk wil je juist dat dit meer flexibeler is en dat je door middel van zo’n containertje de plek kan bepalen van de content.
	(\textbf{Dante Klijn:} In theorie zou je die container ook weer kunnen her gebruiken).

	\whitespace
	Ik denk dat misschien zouden dat een soort van sectie zijn of subsectie en dat jij die kan sorteren binnen je sectie.
	Maar dan ben ik alweer helemaal in mijn hoofd met CMS-termen er zijn genoeg en andere CMSen die het anders noemen.
	Alleen in ieder geval is het positioneren en groeperen van de content erg belangrijk.
}

\DanteInt{Zijn er nu ook problemen waar je nu tegen aanloopt, waarbij je denkt oh ik moet dit weer doen.
	Of ik moet dit scriptje weer runnen om het weer goed te krijgen.}

\RobInt{Voor mij gevoel valt het wel mee ik werk natuurlijk ook al heel lang met het CMS en ben ook meegegroeid met het CMS.
	Dus ik weet ook hoe ik dingen moet inrichten en wat ik wel en niet moet doen.
	Maar ik denk dat ik misschien al automatisch om dingen heen werk.
	Het is wel dat ik er in onze CMS tunnel zit.}

\DanteInt{Moet jij vaak iemand helpen als iemand iets heeft kapot gemaakt.}

\RobInt{Ik richt zelf het CMS in dus maar stijlen en secties toe te voegen.
	Het goed overerven hoe je moet configureren in het tweede weet je dus echt bijvoorbeeld als ze daar een default mapje footer kunnen instellen en inrichten en die hebt over de rest van de catgorie stijlen.
	Dat soort zaken dat richt ik allemaal zelf in en de services desk die weet inmiddels ook wel hoe ik het inrichten en het CMS in het algemeen werkt.
	Ik denk dat het dat soort vragen van klanten in ieder geval meer bij de servicedesk terecht komen en dat die het meestal ook wel op kunnen lossen.}

\DanteInt{Als jij fundamenteel iets aan kon passen aan het CMS. Het zou geen geld of tijd kosten wat zou jij dan anders willen zien, omdat je hier bijvoorbeeld je teen de hele tijd tegen aanstoot.}

\RobInt{Nou het belangrijkste vind ik in ieder geval wel dus die container waar we net al over gesproken hebben dus daar iets mee meer flexibiliteit.
	Ik denk dat artikeltypes al  flexibel genoeg is ieder geval daar hebben we al heel veel mee types aangemaakt met de juiste velden enzo.
    Het URL-beheer het gisteren ook nog met Erwin over dat zou beter kunnen zeg maar maar (\textbf{Dante Klijn}: met die seo urls).
	Je hebt nu een aantal dingen die gewoon vanuit de historie zeg maar mee zijn gekomen.
	Vroeger met het CMS publiceerde je gewoon een  artikel en die kreeg gewoon een url wel dingetje tegenwoordig gebruiken wij een artikel ook vaak als een bouw blokje als onderdeel van een categorie van een pagina ja van een artikel een linkje binnen.
	Bijvoorbeeld dat heb je een card dat is gewoon een elementje daar hoef je helemaal geen link voor te hebben.

	\whitespace
	Een artikel met een volledige weergaven moet een url hebben en degene die geen volledige weergave hebben moeten dat niet.
	Verder staan staat nu de hele url history storie van een artikel staat In de links tabel.
	Met andere woorden Ik heb er nog een artikel over met een detailweergave pas ik de url van aan vervolgens berekend die zelf van hoofd url van het artikel dan past die ook de overige urls aan.

	\whitespace
	Dit was hele logische functionaliteit totdat wij de redirects tabel hebben geïntroduceerd
	Nu heb je op twee plekken urls te beheren namelijk je hebt de url historie allemaal zo heb je mensen die gaan ook al zeggen van hoe ben je wilt URL in de historie toevoegen en van Ik pas hem  aan en weer terug aan  om te opslaan dan voegt hij het toe aan de history tabel tewijl de redirects tabel daar veel beter voor geschikt is.
	De redirecttabel houd ook bij hoe vaak hij is getriggerd en door wie. Hierdoor kan je deze urls om de zoveel tijd opschonen wat nu niet mogelijk is in de links tabel.
	Eigenlijk zou je dan een hoofd url willen hebben en een redirects tabel en dan zijn de links en history tabellen niet meer nodig.
}

\DanteInt{Het CMS is niet alleen maar negatief er zijn waarschijnlijk ook positieve punten.
	Als jij mocht kiezen van wat je graag terug wou zien wat zou dat dan zijn?}

\RobInt{Het groeperen in een containertje waar we het eerder ook al over hebben gehad Dat is wel het voornaamste punt waar ik heel erg mis.
	Verder is de zoekfunctie sup optimaal, zoekresultaten zijn er niet altijd en je kan niet alijd vinden wat je altijd zoekt.
	Dit heeft met het CMS te maken want in het CMS configureer je de artikelen zodat ze geïndexeerd worden of niet dat weer invloed heeft op de zoekresultaten.}

\DanteInt{zijn er nog specifieken nog andere punten die je nog wilt melden}

\RobInt{De sectie mogelijkheden die er nu zijn erg prettig en kan ik nu veel mee.
	Categorie stijlen wel en niet tonen.
	Hierbij is het over erven van de artikelen erg belangrijk.
	Door secties kan je zeggen dat bepaalde content altijd bovenaan of onderaan moet staan bijvoorbeeld de footer en header dat is belangrijk dat nog steeds kan.
}

\DanteInt{Jij werkt ook veel met de database zie je daar ook veel problemen mee als dataanalist?}

\RobInt{
	Het is vooral veel legacy en er zit veel logica in de database.
	De business logica zit nu vooral in de database, in een modern systeem zou je dat heel anders maken want daar stop nouwlijks of geen logica in de database.
	Dan zouden het vrijwel alleen maar tabellen zijn en niet meer tegen stored procedures praten.
    
    \whitespace
    Tabellen zijn nu overbodig of worden niet meer gebruikt.
    Je hebt de beruchte TPL-tabellen en bijvoorbeeld links tabellen.
}


