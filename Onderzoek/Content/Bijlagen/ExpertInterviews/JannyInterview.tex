\section{Janny Reitsma Interview}

\subsection*{Vragen}
\begin{itemize}
    \item{Introductie}
    \begin{itemize}
        \item[1]{Wie ben je?}
        \item[2]{Wat doe je binnen Snakeware?}
    \end{itemize}
    \item{Profiel van de klanten}
    \begin{itemize}
        \item[3]{Wie gebruikt nu het CMS zijn dat mensen in de winkels of systeembeheerders?}
    \end{itemize}\item{Positieve interacties met klanten}
    \begin{itemize}
        \item[4]{Wat vinden klanten nu het fijnst aan het Snakeware Cloud platform?}
        \item[5]{Als er punt was van Snakeware Cloud was waar meer aandacht aan besteed kon worden wat zal dan zijn?}
    \end{itemize}\item{Positieve kanten van het huidige CMS}
    \begin{itemize}
        \item[6]{Waar lopen klanten momenteel het vaakst tegenaan?}
        \item[7]{Als jij iets fundamenteels kon veranderen aan het CMS wat zou dat dan zijn?}
    \end{itemize}
\end{itemize}


\subsection*{Transcriptie}

\DanteInt{misschien een klein voorstel rondje, maar  we kennen elkaar natuurlijk al een beetje.}

\JannyInt{Ik ben Janny Reitsma ik werk al een poosje bij Snakeware al ongeveer 13 jaar.
	Ik werk bij de service desk van Snakware ... hoe het CMS werkt en hoe ze het kunnen gebruiken.
	Ik geef vaak cursussen aan nieuwe klanten als ze een nieuwe site aan nemen van Snakeware.}

\DanteInt{Hoe ziet de huidig klant er nu uit?
Is dat een normale gebruiker of is dat een systeembeheerder.}

\JannyInt{Dat zijn meestal marketing mensen die de content van de site inrichten.
... de marketing afdeling van de kleinere klanten is meestal heel klein en is vaak door dezelfde persoon afgehandeld die andere taken afhandelt in het bedrijf.}

\DanteInt{Je werkt dus veel met de klanten die het CMS gebruiken.
	Wat vinden klanten nu het fijnst van het CMS?
    Misschien andere geformuleerd wat zou willen houden van het huidige CMS?}

\JannyInt{
	Wat ik nu terug krijg is dat mensen door secties verward wordt.
	Maar de categorieën moeten blijven.
	Het is een goede structuur met de blauwe mappen moet blijven.
    Maar met secties die je secties kan in en uit klappen is erg onduidelijk.
	Het is ook niet meer van deze tijd hoe het nu uitgeschreven wordt.
	Hierbij denk ik dat een visueler ingericht kan worden door het CMS zodat de klant weet wat hij zij doet.
	Het huidige CMS is het lastige doen voor de klant wat hij doet.
	Als je nu 3 cards naast elkaar probeert te zetten dan is het nu niet makkelijk te zien wat je doet.
	Dus een van de grotere aanpassingen die dan zou willen zien is dat het aanpassen van de stijl.
	Je heb nu Laptop desktop iPhone dan zou je in het CMS kunnen zien dat het er goed uit ziet in de verschillende layouts.
	Hierbij moet de klant dus makkelijk kunnen zien wat hij of zij doet.
}

\DanteInt{Als je een punt hebt van het CMS wat je graag verbeterd zou willen zien wat zou dat zijn?}

\JannyInt{Ik zou graag willen dat de webapplicatie meer visueel werkt zodat de klant weet wat hij of zij doet.
	Verder wil ik ook dat de naamgeving veranderd want de huidige naamgeving is erg verouderd en is niet logisch.}

\DanteInt{ Een ander punt van het CMS is dat er ook negatieve dingen van het CMS.
	Zijn er ook dingen waar klanten het vaakst tegen lopen?}

\JannyInt{Ik kan er zo even niet op komen.
	... Wat heel vaak voor komt, is dat mensen artikelen maken en dan problemen krijgen met SEO-instelling.
	Ze kunnen dit vaak fout doen en zorgt voor veel belletjes bij de service desk.
	Hier komt het vaak voor dat ze proberen een dubbele URL in te stellen en omdat het artikel in de prullenbak staat dus kunnen ze hem nog niet gebruiken.
}

\DanteInt{Is het dan niet een primair probleem dat SEO in het begin nog niet mee genomen is.
	Omdat het toen nog niet bestond natuurlijk}

\JannyInt{Ja, ik zou wel uit SEO werken want SEO is erg belangrijk en zou ook wel een stukje AI te zien met het schrijven SEO-teksten.}

\DanteInt{Als je zelf iets fundamenteels kon veranderen aan het CMS en het zou geen tijd of geld kosten wat zou dat dan zijn?}

\JannyInt{Ik zou graag willen dat artikelen gemakkelijk kan drag en droppen.
	Je wilt gewoon kunnen drag en dropen zodat je de gemakkelijk kan sorteren.
	Dat je daar voor niet eerst voor een knopje moet drukken om het voor elkaar te krijgen.
	Wat mensen ook vaak missen is dat het nu gelijk live komt te zijn.
	Ik zou ook zeker mee nemen dat je niet direct live werkt maar dat je met concepten werkt.
}

\DanteInt{Dit is het eigenlijk wel ik weet niet of jij nog suggesties of dingen die je wilt bespreken.}

\JannyInt{Het koppelen met externe tooling kost nu vaak tijd om dat voor elkaar te krijgen, maar dat zou de klant zelf kunnen regelen.
	Dit moet nu vaak custom er in zetten en ik denk dat andere CMS dit makkelijk kunnen doen.
	Wij waren er altijd een beetje benauwd voor, maar de kennis van onze klanten gaat ook verder en is wel een grotere wens dat dit makkelijker gaat.}

\DanteInt{Dus het moet geen custom werk zijn maar er moet een oplossing voor zijn in het CMS.}

\JannyInt{Het kost nu 2 uur werk deze uren moeten nu geboekt worden en kosten de klant geld.
	Als de klant dit zelf kon doen kostte het hun 10 minuten.
	Verder is het ook belangrijk dat de data wordt gemeten waar de acties op de sites worden gemaakt.
	Dit soort dingen moeten ze zelf kunnen instellen zodat Snakeware een goedkopere partij van.
    Een ander punt is dat de klant soms niet begrijpt waar deze kosten vandaan komen.
}
