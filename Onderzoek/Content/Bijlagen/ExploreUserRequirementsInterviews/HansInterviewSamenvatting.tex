\section{Hans Hoomans interview}
\label{appendix:ExploreUserRequirementsHans}

\subsection*{Vragen}
\begin{itemize}
    \item{Introductie}
    \begin{itemize}
        \item[1]{Wie ben je?}
        \item[2]{Wat doe je binnen Snakeware?}
    \end{itemize}
    \item{Applicatie}
    \begin{itemize}
        \item[3]{Wat zijn de kern functionaliteiten die de klant moet kunnen doen binnen het systeem?}
        \item[4]{Zijn er specifieke eisen en wensen die je graag in het proof of concept wilt zien?}
        \item[5]{Wat is jouw visie voor een nieuw CMS-systeem?}
        \item[6]{Wat is de prioriteit van SEO-integratie in het proof of concept?}
        \item[7]{Als de prioriteit hoog is hoe zie je dat dan voor je?}
        \item[8]{Wat is jouw visie van artikel types en formulieren binnen het proof of concept?}
    \end{itemize}
\end{itemize}


\subsection*{Transcriptie}

Tijdens het interview met hans is er vooral veel gesproken over een theoretisch datamodel dat Hans heeft bedacht.
Dit datamodel heeft hij al een lange tijd in zijn hoofd en zou graag het gerealiseerd willen zien.

\whitespace
Het is wel duidelijk gemaakt dat ik niet dit datamodel zal implementeren tenzij het de goede oplossing zou zijn of dat er modificaties op het model worden gedaan.
Dit model is meer gericht op generaliseerde data inricht en hoeft niet per se met het CMS samen te werken.
Het model werkt door gebruik te maken van collecties, een collectie is een value of een andere collectie.

\whitespace
Het idee voor het model kwam door de gedacht van dat als het simpel wordt opgelost het makkelijker is om te onderhouden.
Hier kwam Hans zijn quote van daan \qw{De kracht van eenvoud en herhaling}.
Hierom wilt hans gebruik maken van een soort geneste datamodel zodat het vooral herhaald wordt en niet onnodige tabbellen aangemaakt moeten worden.
Want dat is nu een groot pijnpunt van hans is dat het nu zo erg complex gemaakt is door de hoeveelheid relaties en tabellen waar het CMS meet interacteerd.

\whitespace
Verder wilt Hans graag gebruik maken van cashing en NoSQL databases.
Dit zijn moderne technieken en denk hier een grote performance winst uit te halen.

\whitespace
Vertalingen doen op sites is nu een zwaar punt bij Snakeware en had in zijn datamodel in gedachten dat dit er in gebouwd zou moeten worden.
Snakware Cloud ondersteund niet vertalingen out of the box, het ondersteunen meerder sites (dus een site per taal) maar niet meertaligheid in een site.
Hoe Snakeware he nu soms op pakt, is dat de talen in de site zijn code gezet moeten worden dit kost veel tijd en daar bij ook geld voor Snakeware.
Dit zou hans graag willen zien in een nieuw concept van het CMS dat het in het datamodel ingebouwd is. 

\whitespace
Een van de mogelijke toekomst wensen is dat het nieuwe CMS een drag en drop frontend heeft.
Dit zou door het nieuwe datamodel mogelijke worden gemaakt dat het goed te implementeren is.

