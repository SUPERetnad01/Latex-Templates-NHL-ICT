\section{Betrouwbaarheid \& Validiteit}
Tijdens het onderzoek is er gebruikgemaakt van \textit{ICT Research methodes} om de betrouwbaarheid van de onderzoeksmethoden te verhogen.
Elke deelvraag heeft gebruikgemaakt van een of meerdere onderzoeksmethoden.
Alle resultaten van de deelvraag zijn terug gekoppeld met de personen die betrokken waren tijdens het beantwoorden van de deelvragen in het onderzoek.
Dit is gedaan om de validiteit van het onderzoek te verhogen.

\whitespace
Bij de verschillende deelvragen is er gebruikgemaakt van gekwalificeerde medewerkers van Snakeware.
Tijdens stakeholder analyse is er gebruikgemaakt van de product owner Elsa Croes.
Zij weet veel van de kleine bedrijven van Snakeware en weet goed welke partijen belangrijk zijn voor het \qw{Het CMS voor iedereen} project.
Tijdens de IT architecture sketching is gebruikgemaakt van de architect van het CMS Erwin Keuning en de daarna meest ervaren developer van het CMS Kevin Snijder.

Bij de expert interviews is er gebruikgemaakt van Janny Reitsma en Rob Douma twee experts op hun vakgebied.
Jany Reitsma handelt de communicatie tussen de klanten en het R\&D team.
Rob Douma is een expert gebruiker van het CMS, hij richt veel omgevingen in en heeft veel technische kennis van het CMS.

Tijdens de explore user requirements interviews is er gebruikgemaakt van Elsa Croes en Hans Hoomans.
Elsa Croes is zoals eerder vermeld erg bekend met de huidige kleine klanten van Snakeware.
Om haar te ondersteunen is Eric Dijkstra aangeschoven, een frontend developer die vaak taken doet voor de kleinere klanten.
Hans Hoomans is de CEO van Snakware en representeert de toekomstvisie van het bedrijf.
%
% \todo[inline]{Waarom is het onderzoek \qw{Goed}}
% \todo[inline]{Benoem de verschillende onderzoeksmethoden en het impact er van}
