\section{Eindproduct}
Om de gestelde doelen te bereiken, zullen er vier eindproducten worden ontwikkeld.
Deze eindresultaten omvatten het projectresultaat, dat aan het einde van de afstudeerperiode wordt gepresenteerd en gedemonstreerd.
De vier producten in kwestie zijn: het Plan van Aanpak, het Onderzoeksverslag, het Technisch Verslag en het Eindproduct.

\whitespace[2]
\textbf{Plan van Aanpak}: Dit document beschrijft in detail de uitvoering van de opdracht.

\whitespace[2]
\textbf{Onderzoeksverslag}: In het onderzoeksverslag staat het uitgewerkte onderzoek.
Het onderzoek zal een lijst van geprioriteerde requirements opleveren die gebruikt worden tijdens de ontwerpfase en realisatiefase.

\whitespace[2]
\textbf{Technisch verslag}: In het Technisch Verslag worden de beslissingen die zijn genomen tijdens het uitvoeringsproces duidelijk uiteengezet, en wordt het ontwerp gepresenteerd met onderbouwing van de gemaakte keuzes.
Voor het opstellen van zowel het ontwerp als de implementatie is het van cruciaal belang om rekening te houden met de eisen en wensen die zijn voortgekomen uit de requirementanalyse. \\
Tijdens het ontwerpen wordt er gebruikgemaakt van het 4 + 1 model, waarin de architectuur van het systeem duidelijk zichtbaar wordt.
Verder dient het technisch verslag gedetailleerd te beschrijven hoe het systeem functioneert, inclusief de keuzes die zijn gemaakt op basis van het ontwerp.
Om het systeem door loops te testen wordt er gebruikgemaakt van het V-Model \Parencite{VModel}.
Aan het einde van het technisch verslag wordt er gereflecteerd op de ontwerp- en het realisatieproces, waarbij er gebruikgemaakt wordt van de STARR-methode \Parencite{STARR}.

\whitespace[2]
\textbf{Product}: Het product is een proof of concept \gls{CMS}-API die de data kan tonen op een simpele frontend.
Bekijk hoofdstuk \ref{sec:Opdrachtomschrijving} voor meer informatie.
% Het product omvat de uitwerking van het proofkan tonen of concept. Zie meer informatie hoofdstuk 1.4 en het opleveren van de broncode naar Snakeware.
